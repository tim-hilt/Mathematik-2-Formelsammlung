\message{ !name(Mathe2Formelsammlung.tex)}\documentclass[8pt, a4paper]{article}

\usepackage[german]{babel}
\usepackage[utf8]{inputenc}
\usepackage{mathtools}
\usepackage[colorlinks=true,linkcolor=black]{hyperref}
\usepackage{xcolor}
\usepackage{sectsty}

\sectionfont{\color{red}}
\subsectionfont{\color{red}}

\everymath{\displaystyle}

\title{Mathematik 2 Formelsammlung}
\author{Tim Hilt}
\date{23. März 2018}

\begin{document}

\message{ !name(Mathe2Formelsammlung.tex) !offset(-3) }


\maketitle
\pagebreak

\tableofcontents
\pagebreak

\setcounter{section}{13}
\section{Differenzialgleichungen Erster Ordnung}

\subsection{Separierbare Differenzialgleichung}
Eine Differenzialgleichung Erster Ordnung, die man in der Form

\[
  y'=\frac{f(x)}{g(y)}
\]

schreiben kann, bezeichnet man als \textbf{separierbar}

\subsection{Separation der Variablen}

Die allgemeine Lösung einer separierbaren Differenzialgleichung kann man durch folgende Schritte bestimmen
\begin{enumerate}
\item Ersetze \(y'\) formal durch \(\frac{dy}{dx}\).
  \item Separiere alle Terme in \(x\) und alle Terme in \(y\) und bringe die Differenzialgleichung damit in die Form \(g(y)dy=f(x)dx\).
  \item Integriere symbolisch \(\int{g(y)dy} = \int{f(x)dx}\) separat auf beiden Seiten.
  \item Löse die integrierte Gleichung nach der gesuchten Funktion \(y(x)\) auf.
\end{enumerate}

\subsection{Lösungsansatz homogene Dgl 1. Ordnung}

Die allgemeine Lösung \(y_h\) einer homogenen linearen Differenzialgleichung

\[
  a_1(x)y'+a_0(x)y=0
\]

erster Ordnung lässt sich durch Separation bestimmen und lautet

\[
  y_h(x)=Ce^{-\int{\frac{a_o(x)}{a_y(x)}dx}}
\]

\subsection{Variation der Konstanten}

Eine partikuläre Lösung einer linearen Differenzialgleichung erster Ordnung

\[
  a_1(x)y'+a_0(x)y=r(x)
\]

lässt sich durch \textbf{Variation der Konstanten} bestimmen:

\begin{enumerate}
\item Berechne die allgemeine Lösung der homogenen Differenzialgleichung.
\item Ersetze die Konstante \(C\) in der homogenen Lösung durch eine Funktion \(C(x)\).\\
  Daraus ergibt sich ein Ansatz \(y_p\) für eine partikuläre Lösung.
\item Bestimme die Funktion \(C(x)\) durch Einsetzen von \(y_p\) in die Differenzialgleichung.
\end{enumerate}

\subsection{Nichtlineare Differenzialgleichungen}

Nichtlinear ist eine Differenzialgleichung dann, wenn sie Produkte ihrer Lösung \(y(x)\) oder der Ableitungen beinhaltet\\
Beispiele:

\[
  \begin{split}
    % split bewirkt hier, dass die \\ auch als Zeilenumbruch gewertet werden!
    y''*y=3x\\
    2y'*y^2=0\\
    y'*\frac{y}{x}=4x^2
  \end{split}
\]

\subsection{Exponentialansatz}

\[
  y(x)=e^{\lambda x}
\]

\end{document}

%%% Local Variables:
%%% mode: latex
%%% TeX-master: t
%%% End:
\message{ !name(Mathe2Formelsammlung.tex) !offset(-109) }
