\documentclass[12pt, a4paper]{scrreprt}

\usepackage{mystyle}
\usepackage{float}
\usepackage{makecell}
\usepackage{lipsum}

\begin{document}

\newgeometry{margin=1.5cm}
\begin{titlepage}

  \includesvg[width=0.25\textwidth]{Grafiken/LogoHS_Esslingen}\\ \vspace{3cm}
  
  \begin{center}
    {\usekomafont{disposition}
      \Huge Formelsammlung Mathematik 2}
    \vspace{0.5cm}
    
    \begin{Large}
      Tim Hilt\\
      \vspace{0.4cm}
      \today\\
    \end{Large}
    
  \end{center}
\end{titlepage}
\restoregeometry

%%% Local Variables:
%%% mode: latex
%%% TeX-master: "Mathe2Formelsammlung"
%%% End:


\tableofcontents
\pagebreak

\setcounter{chapter}{13}
\chapter{Differenzialgleichungen}

\section{Differenzialgleichungen erster Ordnung}

\subsection{DGL aus Richtungsfeld bestimmen}
Eine Differenzialgleichung erster Ordnung beschreibt immer die Steigung einer Funktion. Am einfachsten ist es in diesem Fall, sich zu überlegen, wann die Steigung \(=0\) wird; also für \(y'(x)\) 0 einzusetzen und sich dann zu überlegen, wann die Gleichung erfüllt ist; wo also die Steigung \(=0\) wird.

Beispiel:\\[1em]
\makebox[3cm]{\(y'(x) = xy\) \hfill} wird null bei \(x = 0\) oder \(y = 0\).\\
\makebox[3cm]{\(y'(x) = x + y\) \hfill} wird null, wenn \(x = -y\) ist.\\
\makebox[3cm]{\(y'(x) = x^2\) \hfill} wird null, für \(x = 0\).\\
\makebox[3cm]{\(y'(x) = y^2\) \hfill} wird null, für \(y = 0\).

\subsection{Separation der Variablen}

Die allgemeine Lösung einer separierbaren Differenzialgleichung kann man durch folgende Schritte bestimmen
\begin{enumerate}
\item Ersetze \(y'\) formal durch \(\frac{dy}{dx}\).
\item Separiere alle Terme in \(x\) und alle Terme in \(y\) und bringe die Differenzialgleichung damit in die Form \(g(y)dy=f(x)dx\).
\item Integriere symbolisch \(\int{g(y)dy} = \int{f(x)dx}\) separat auf beiden Seiten.
\item Löse die integrierte Gleichung nach der gesuchten Funktion \(y(x)\) auf.
\item Falls Anfangswertproblem gegeben (Bspw.\ sei \(y(0) = 1\)): Für \(x = 0\) setzen; für \(y = 1\) setzen
\item Nach \(C\) auflösen
\item Gelöste DGL mit neuem C nochmals hinschreiben
\end{enumerate}

\subsection{Lösungsansatz homogene Dgl 1. Ordnung}

Die allgemeine Lösung \(y_h\) einer homogenen linearen Differenzialgleichung

\[
  a_1(x)y'+a_0(x)y=0
\]

erster Ordnung lässt sich durch Separation bestimmen und lautet

\[
  y_h(x)=Ce^{-\int{\frac{a_o(x)}{a_y(x)}dx}}
\]

\subsection{Variation der Konstanten}

Eine partikuläre Lösung einer \textbf{linearen Differenzialgleichung \textcolor{red}{erster Ordnung}}

\[
  a_1(x)y'+a_0(x)y=r(x)
\]

lässt sich durch \textbf{Variation der Konstanten} bestimmen:

\begin{enumerate}
\item Berechne die allgemeine Lösung der homogenen Differenzialgleichung (meist durch Separation der Variablen)
\item Ersetze die Konstante \(C\) in der homogenen Lösung durch eine Funktion \(C(x)\).\\
  Daraus ergibt sich ein Ansatz \(y_p\) für eine partikuläre Lösung.
\item Bestimme die Funktion \(C(x)\) durch Einsetzen von \(y_p\) in die Differenzialgleichung.
\end{enumerate}

\begin{framed}
  \textbf{Achtung: Signalcharakter!!! Bei DGL erster Ordnung mit Störfunktion meist zuerst Separation der Variablen, dann Variation der Konstanten!!!}
\end{framed}

\subsubsection{Beispiel}

DGL:\ \(y'+\frac{1}{x}y=\frac{2}{1+x^2}\)

\begin{enumerate}
\item zuerst homogene Lösung finden
  \[\rightarrow y_h(x)=\frac{C}{x}\]
\item Variation der Konstanten; ersetze \(C\) durch \glqq{}Pseudofunktion\grqq{} \(\textcolor{green}{C(x)}\):
  \[y(x)=\frac{\textcolor{green}{C(x)}}{x}\]
\item Berechne alle relevanten Ableitung der Funktion \(y(x)\) und setze in ursprüngliche \textbf{inhomogene} DGL ein:

  \begin{gather*}
    \intertext{Ableiten:}
    y(x)=\frac{\textcolor{green}{C(x)}}{x}\\
    y'(x)=\frac{\textcolor{green}{C'(x)}*x-\textcolor{green}{C(x)}}{x^2}\\
    \intertext{Einsetzen:}
    \underbrace{\frac{\textcolor{green}{C'(x)}*x-\textcolor{green}{C(x)}}{x^2}}_\text{\(y'(x)\)}+\frac{1}{x}*\underbrace{\frac{\textcolor{green}{C(x)}}{x}}_\text{\(y(x)\)}=\frac{2}{1+x^2}
  \end{gather*}

\item Soweit als möglich kürzen \textbf{(\(C(x)\) muss sich immer kürzen!!)}
\item Nach \(C'(x)\) auflösen und aufleiten
\item Beim Aufleiten hier \textbf{Integrationskonstante \(C\) weglassen}, da ja ein Wert für C gesucht ist!
\item \textbf{\(C(x)\)} wieder zurück \textbf{einsetzen in partikulären Ansatz} \(y_p(x)\) und soweit als möglich kürzen
\item Allgemeine Lösung \(y(x)\) ergibt sich aus Addition von \(y_h(x)\) und \(y_p(x)\)
  \[y(x)=y_h(x)+y_p(x)\]
\item Bei AWP jetzt noch Wert am gegebenen Punkt einsetzen
  
\end{enumerate}
\section{Differenzialgleichungen höherer Ordnung}

\subsection{Allgemeine Lösung einer Dgl}

Die allgemeine Lösung einer linearen, homogenen Dgl \(n\)-ter Ordnung mit konstanten Koeffizienten wird in der Form

\[
  y(x) = C_1 e^{\lambda_1 x} + C_2 e^{\lambda_2 x} + C_3 e^{\lambda_3 x} + \dots + C_n e^{\lambda_n x}
\]

geschrieben. Jedes \(e^{\lambda_k x}\) ist dabei eine Fundamentallösung.\\
\textbf{Beachte:} Formelparameter ist nicht immer \(x\)!

\subsection{Nichtlineare Differenzialgleichungen}

Nichtlinear ist eine Differenzialgleichung dann, wenn sie \mytextred{\textbf{Produkte ihrer Lösung \(y(x)\) oder der Ableitungen}} beinhaltet\\[1em]
Beispiele:

  \begin{align*}
    % split bewirkt hier, dass die \\ auch als Zeilenumbruch gewertet werden!
    y''*y &=3x\\
    2y'*y^2 &=0\\
    y'*\frac{y}{x} &= 4x^2
  \end{align*}

\subsection{Lineare DGL n-ter Ordnung lösen}

\begin{enumerate}
\item Berechne die allgemeine Lösung \(y_h(x)\) der homogenen Gleichung
\item Berechne eine partikuläre Lösung \(y_p(x)\) der inhomogenen Differenzialgleichung
\item Die allgemeine Lösung einer \textbf{inhomogenen} linearen DGL \(y(x)\) ergibt sich aus der Addition der homogenen Lösung \(y_h(x)\) und einer partikulären Lösung \(y_p(x)\)
\end{enumerate}

\[
  y(x)=y_h(x)+y_p(x)
\]

Bei \textbf{homogenen} DGLs Hergang derselbe, nur eben ohne partikulären Ansatz!

\subsection{Charakteristische Gleichung}

Zur homogenen linearen Differenzialgleichung

\[
  a_n y^{(n)}-a_{n-1}y^{(n-1)}+ \cdots +a_2y''+a_1y'+a_0y=0
\]

gehört die \textbf{charakteristische Gleichung}

\[
  a_n\lambda^n+a_{n-1}\lambda^{(n-1)}+ \cdots + a_2\lambda^2+a_1\lambda+a_0=0
\]

\subsection{Lösungsansatz für homogene, lineare DGLs mit AWP}

Bei DGL \mytextred{\(n\)-ter} Ordnung sind auch \(\mytextred{n}\) Anfangswerte gegeben!

\begin{enumerate}
\item Charakteristische Gleichung der DGL erstellen
\item Eigenwerte herausfinden
\item Allgemeine Lösung erstellen
\item Allgemeine Lösung \mytextred{\(n\)} mal ableiten
\item Die Anfangswerte einsetzen und so alle Konstanten bestimmen
\item Neue, spezielle Lösung mit den zuvor bestimmten Konstanten formulieren
\end{enumerate}

\subsection{Fälle beim Lösen von Eigenwerten}

\subsubsection{Mehrfache reelle Eigenwerte}

Wenn \(\lambda\) ein doppelter Eigenwert (z.B. MNF ergibt zweimal 2 oder so) der DGL ist, so werden trotzdem zwei reelle Fundamentallösungen erzeugt:

\[
  y_1(x)=e^{\lambda x}, y_2(x)=xe^{\lambda x}
\]

Je höher die Vielfachheit der Nullstelle, umso größer die Potenz auf dem vorangestellten \(x\). Zum Beispiel bei vierfacher Nullstelle:

\[
  \begin{split}
    y_1(x)=e^{\lambda x},\\
    y_2(x)=xe^{\lambda x},\\
    y_3(x)=x^2e^{\lambda x},\\
    y_4(x)=x^3e^{\lambda x}
  \end{split}
\]

\subsubsection{komplexe Eigenwerte}

Jedes konjugiert komplexe Paar Eigenwerte \(\lambda_{1,2}=\textcolor{red}{a} \pm i\textcolor{green}{b}\) erzeugt zwei Fundamentallösungen:

\[
  y_1(x)=e^{\textcolor{red}{a}x}\cos(\textcolor{green}{b}x),\qquad y_2(x)=e^{\textcolor{red}{a}x}\sin (\textcolor{green}{b}x)
\]


\subsection{Störansatz}

Bei linearen \textbf{DGLs erster Ordnung} kann eine partikuläre Lösung mithilfe von \textit{Variation der Konstanten} ermittelt werden.\\
Dieser Ansatz funktioniert jedoch bei DGLs höherer Ordnung nicht mehr! Bei DGLs höherer Ordnung wird die partikuläre Lösung anhand eines \textbf{Störansatzes} ermittelt. Dieser ergibt sich aus der Art der Störfunktion. Für verschiedene Arten von Funktionen gibt es verschiedene Störansätze. Diese können durch nachschlagen in einer \textbf{Störansatztabelle} ermittelt und eingesetzt werden.\\[1em]

\begin{center}
  \makegapedcells{}
  \begin{tabular}{l | l | r}

    Störfunktion & Ansatz für partikuläre Lösung\\
    \hline \hline
    \makecell{Polynom vom Grad \mytextred{n}:\\ \(r(x) = a_0 + a_1x + a_2 x^2 \dots a_nx^{\mytextred{n}}\)} & \makecell{Polynom vom Grad \mytextred{n}\\\(y_p(x) = A_0 + A_1x + A_2 x^2 \dots + A_n x^{\mytextred{n}}\)}\\ \hline
    \makecell{Exponentialfunktion\\ \(r(x) = ae^{\mytextred{k}x}\)} & \makecell{Exponentialfunktion\\ \(y_p(x) = Ae^{\mytextred{k}x}\)}\\ \hline
    \makecell{Harmonische Schwingung\\ \(r(x) = a_1 \cos (\mytextred{\omega} x) + a_2 \sin (\mytextred{\omega} x)\)} & \makecell{Harmonische Schwingung\\ \(y_p(x) = A_1 \cos (\mytextred{\omega} x) + A_2 \sin (\mytextred{\omega} x)\)}\\ \hline
    \makecell{Gedämpfte harmonische Schwingung\\ \(r(x) = e^{\mytextred{k}x} (a_1 \cos (\mytextred{\omega} x) + a_2 \sin (\mytextred{\omega} x))\)} & \makecell{Gedämpfte harmonische Schwingung\\ \(y_p(x) = e^{\mytextred{k} x} (A_1 \cos (\mytextred{\omega} x) + A_2 \sin (\mytextred{\omega} x))\)}\\ \hline
  \end{tabular}
\end{center}

\subsubsection{Vorgehen bei partikulärer Lösung mit Störansatz}

\begin{enumerate}
\item Homogene Lösung \(y_h(x)\) lösen
\item Resonanz prüfen
\item geeigneten Störansatz wählen und Störansatz ableiten, bis die höchste auftretende Ableitung erreicht wurde
\item Störansatz + Ableitungen in ursprüngliche \textbf{inhomogene} DGL einsetzen
\item alles ausmultiplizieren und nach Potenzen ordnen
\item Koeffizientenvergleich mit der \glqq{} originalen \grqq{} Störfunktion (Faktoren der gleichen Potenz werden gleichgesetzt und aufgelöst)
\end{enumerate}

\subsubsection{Superposition}

Ist die Störfunktion \(r(x)\) eine zusammengesetzte Funktion von Addition oder Subtraktion verschiedener Einzelfunktionen \(r(x)=r_1(x)\pm r_2(x)\pm \cdots r_n(x)\), so kann für \textbf{jede Einzelfunktion der Störansatz separat errechnet werden}. Dieses Vorgehen nennt man \textbf{Superposition}.

\subsubsection{Resonanz}

Ist die \textbf{gesamte} Störfunktion (oder bei Superposition eine der Störfunktionen) in der allgemeinen Lösung der homogenen DGL enthalten, so liegt \textbf{Resonanz} vor.\\
In einem solchen Fall muss der für den Typ der Störfunktion gewählte Ansatz mit \(x\) multipliziert werden. Ist der resonante Eigenwert ein mehrfacher (\(n\)-facher) Eigenwert, so wird der Ansatz mit \(x^n\) multipliziert.

\subsection{Eulerverfahren}

Um das Eulerverfahren anwenden zu können benötigen wir eine Differenzialgleichung 1. Ordnung, sowie einen Funktionswert an der Stelle \(f(x_0) = m\) und eine Schrittweite \(h\)\\[1em]
Algorithmus für das Vorgehen:


\begin{align*}
  y_{k+1} &= y_k + h * y'(x_k)\\
  x_{k+1} &= x_k + h
\end{align*}



\textbf{Vorgehen:}
\begin{enumerate}
\item Differenzialgleichung nach \(y'(x)\) auflösen
\item Startwerte \(x_0\) und \(y_0\) festlegen (sind gegeben)
\item \(y_1\) berechnen: \(y_1 = y_0 + h * y'(x_0)\)
\item \(x_1 = x_0 + h\)
\item \(y_2\) berechnen: \(y_2 = y_1 + h * y'(x_1)\)
\item \(x_2 = x_1 + h\)
\item \dots
\end{enumerate}

\textbf{Beispiel:}

\glqq{} Berechnen Sie zwei Schritte des Eulerverfahrens für die DGL \(y'(x) + 4x^3 \cdot y(x) = 0, \quad y(1)= 1,\quad h = 0.1\) \grqq{}

\begin{enumerate}
\item Nach \(y'(x)\) auflösen: \(y'(x) = -4x^3 \cdot y(x)\)
\item Startwerte festlegen: \(x_0 = 1,\quad y_0 = 1\)
\item \(y_1\) berechnen: \(y_1 = y_0 + h \cdot y'(x_0) \quad = 1 + 0.1 \cdot (-4\cdot1^3 \cdot 1) = \dots\)
\item \(x_1\) berechnen: \(x_1 = x_0 + h = 1 + 0.1 = 1.1\)
\item \(y_2\) berechnen: \dots
\end{enumerate}

\section{Differenzialgleichungssysteme}

\subsection{Lösung eines Dgl-Systems \(n\)-ter Ordnung}

\[
  \mathbf{z} = C_1 e^{\lambda_1 x}*EV_1 + C_2 e^{\lambda_2 x}*EV_2 + \dots + C_n e^{\lambda_n x}*EV_n
\]

\textbf{Achtung:} Beachte andere Schreibweisen! z.B. Zeilenschreibweise. $\rightarrow$ nach Berechnung wieder in alte Schreibweise bringen!

\subsection{Differenzialgleichung in Differenzialgleichungssystem umschreiben}

Eine inhomogene Differenzialgleichung mit konstanten Koeffizienten

\[
  a_ny^{n} + a_{n-1}y^{n-1} + \cdots + a_2y'' + a_1y' + a_0y = r(t)
\]

kann auch als Differenzialgleichungssystem der Form

\[
  \begin{pmatrix}
    z_1\\
    z_2\\
    z_3\\
    \vdots\\
    z_n
  \end{pmatrix} ' =
  \begin{pmatrix}
    0 & 1 & 0 & \dots & 0\\
    0 & 0 & 1 & \dots & 0\\
    \vdots & \vdots & \vdots & \ddots & \vdots\\
    0 & 0 & 0 & \dots & 1\\
    - \frac{a_0}{a_n} & - \frac{a_1}{a_n} & - \frac{a_2}{a_n} & \dots & - \frac{a_{n-1}}{a_n}
  \end{pmatrix} *
  \begin{pmatrix}
    z_1\\
    z_2\\
    z_3\\
    \vdots\\
    z_n
  \end{pmatrix} +
  \begin{pmatrix}
    0\\
    0\\
    0\\
    \vdots\\
    \frac{r(t)}{a_n}
  \end{pmatrix}
\]

Geschrieben werden.\\[2em]
\myhspace\mytextred{Falls \(n - 1\) Anfangsbedingungen gegeben sind werden diese als Lösungsvektor an der Stelle \(x_0\) geschrieben!}
\begin{figure}[H]
Bsp.:

\textbf{Gegeben}:

\[
  7y''' + 3y'' + 2y' - 5y = \sin(x)
\]
\textbf{Anfangsbedingungen}: \(y(0) = 1, y'(0) = 4, y''(0) = 0\)\\[2em]

\textbf{Gesucht}: Dgl-System mit Lösungsvektor

\textbf{Lösung}:

\[
  \begin{pmatrix*}[c]
    z_1\\
    z_2\\
    z_3
  \end{pmatrix*} ' =
  \begin{pmatrix*}[c]
    0 & 1 & 0\\
    0 & 0 & 1\\
    \frac{5}{7} & - \frac{2}{7} & - \frac{3}{7}
  \end{pmatrix*}
  \begin{pmatrix*}[c]
    z_1\\
    z_2\\
    z_3
  \end{pmatrix*} +
  \begin{pmatrix*}[c]
    0\\
    0\\
    \frac{\sin(x)}{7}
  \end{pmatrix*}
\]
\(\mathbf{z(0)} = \left( \begin{smallmatrix} 1 \\ 4 \\ 0 \end{smallmatrix} \right)\)
\end{figure}

\subsection{Transformiere DGL in DGL-System mit Zustandsvariablen}

Eine DGL \(n\)-ter Ordnung kann mithilfe von \(n\) Zustandsvariablen in ein DGL-System 1. Ordnung umgeschrieben werden.\\[1em]
Dabei ist: {\hskip 5.5cm} Also folgt:

\begin{minipage}{0.45\textwidth}
  \begin{align*}
    z_1 &= y\\
    z_2 &= y'\\
    z_3 &= y''\\
    \vdots\\
    z_n &= y^{(n)}\\
  \end{align*}
\end{minipage}
\hfill
\vline{}
\hfill
\begin{minipage}{0.45\textwidth}
  \begin{align*}
    z_1' &= z_2\\
    z_2' &= z_3\\
    z_3' &= z_4\\
    \vdots\\
    z_n' &= \text{\dots}\\
  \end{align*}
  \(z_n'\): Alles nach der höchsten Ableitung aufgelöst und mit Zustandsvariablen ersetzt
\end{minipage}

\subsection{Kriterien für stabile und instabile Systeme}
\begin{enumerate}
\item Realteile aller Eigenwerte sind negativ: \textbf{System ist asymptotisch stabil}
\item Mindestens ein Eigenwert ist positiv: \textbf{System ist instabil}
\item Mindestens ein Realteil eines Eigenwerts ist 0: \textbf{System ist grenzstabil}
\end{enumerate}

\subsection{Vorgehen bei komplexen Eigenwerten}

Besitzt das DGL-System als Eigenwert ein Paar konjugiert komplexer Eigenwerte \(\lambda_{1/2} = a \pm \mytextred{i}b\), so genügt es, zu \textbf{einem der beiden Eigenwerte} einen Eigenvektor zu berechnen.\\
Der so entstandene komplexe Eigenvektor lässt sich zerlegen in Real- und Imaginärteil mithilfe des eulerschen Satzes: \(e^{(a+\mytextred{i}b)t} = e^{at} \cdot e^{\mytextred{i} bt} = e^{at} (\cos (bt) + \mytextred{i} \sin (bt))\)\\
Bsp.:

\begin{align*}
  z_1(t) &= e^{(3 + 2 \mytextred{i}) t}
  \begin{pmatrix}
    -2 \mytextred{i}\\
    1
  \end{pmatrix}\\
  &= e^{3t} (\cos (2t) + \mytextred{i} \sin (2t))
  \begin{pmatrix}
    -2 \mytextred{i}\\
    1
  \end{pmatrix}\\
         &= e^{3t} \left(
           \begin{pmatrix}
             2 \sin (2t)\\
             \cos (2t)
           \end{pmatrix}
  +\mytextred{i}
  \begin{pmatrix}
    -2 \cos (2t)\\
    \sin (2t)
  \end{pmatrix}
           \right)
\end{align*}

Das DGL-System hat somit die allgemeine Lösung:
\[
  \mathbf{x}(t) = C_1 e^{3t}
  \begin{pmatrix}
    2 \sin (2t)\\
    \cos (2t)
  \end{pmatrix}
  + C_2 e^{3t}
  \begin{pmatrix}
    -2 \cos (2t)\\
    \sin (2t)
  \end{pmatrix}
\]

\subsection{Lösungsstrategie für lineare DGL-Systeme mit konstanten Koeffizienten}

\begin{enumerate}
\item DGL System in Matrixform bringen
\item Eigenwerte der A-Matrix berechnen (Hauptdiagonale -\(\lambda\), Determinante berechnen, charakteristische Gleichung aufstellen, mit 0 gleichsetzen und auflösen)
\item Eigenvektoren aus Eigenwerten berechnen (zu jedem Eigenwert \(\lambda_1\cdots\lambda_n\) ein LGS ausfstellen (\(\lambda\) jeweils einsetzen und neue Matrix \(=0\) setzen) und auflösen)
\item Allgemeine Lösung ist \(y(x)=C_1*e^{\lambda_1*t}*EV_1+\cdots+C_n*e^{\lambda_n*t}*EV_n\)
\item Bei einem AWP, z.B. \(z(0) =
  \begin{smallmatrix}
    n\\
    k
  \end{smallmatrix}
\) Wird \(0\) für \(t\) gewählt und die Lösungsgleichung mit dem Vektor \(
\begin{smallmatrix}
  n\\
  k
\end{smallmatrix}
\) gleichgesetzt. nun werden die Konstanten \(C_1, C_2 \dots C_n\) aus dem entstehenden LGS bestimmt.
\end{enumerate}
Auch hier ist \(y(x)\) eine Summe aus den einzelnen Fundamentallösungen \(y_1\cdots y_n)\)

\subsection{Eulerverfahren bei DGL-Systemen}
\textbf{To do!}

\setcounter{chapter}{9}
\chapter{Potenzreihen}

\section{Reihe}
Eine Reihe ist definiert als die \textbf{Summe einer Folge}, also als:

\[
  \sum_{k=0}^{\infty}a_k  \qquad =  a_0 + a_1 + a_2 + a_3 \cdots
\]

\section{Partialsumme}

Wenn man bei einer Reihe die Summe der ersten \textit{fünf} Reihenglieder bildet nennt man das die \textit{vierte} \textbf{Partialsumme} der Reihe \((a_k)\)

\section{Geometrische Reihe}

Die \textbf{Geometrische Reihe} wird dargestellt durch

\[
  \sum_{k=0}^{\infty} x^k \quad = 1+q+q^2+q^3+ \cdots +q^n \quad = \quad \frac{1}{1-q} \qquad , r = 1
\]

Die geometrische Reihe konvergiert nur für Werte \(|q| < 1\). Für alle anderen Werte für \(q\) divergiert die Reihe.

\section{Konvergenzkriterium}

Eine Reihe konvergiert nur dann, wenn ihre Glieder eine \textbf{Nullfolge} bilden; wenn Sie gegen Null streben.

\section{Reihe der e-Funktion \(e^x\)}

Die \textbf{Reihe der e-Funktion} wird dargestellt durch

\[
  \sum_{k=0}^{\infty}\frac{x^k}{k!} \quad = 1 + \frac{x}{1!} + \frac{x^2}{2!} + \frac{x^3}{3!} \cdots \frac{x^n}{n!} \qquad , r = \infty
\]

\section{Potenzreihe (mit Entwicklungspunkt)}

Eine Potenzreihe mit dem \textbf{Entwicklungspunkt} \(x_0\) wird nach dem Muster:

\[
  p(x) = \sum_{k=0}^{\infty} a_k(x- \mathbf{x_0} )^k \quad = a_0 + a_1(x-x_0) + a_2(x-x_0)^2 + \cdots
\]

gebildet.

Falls \(x_0=0\) ergibt sich dementsprechend:

\[
  p(x) = \sum_{k=0}^{\infty} a_k x^k \quad = a_0 + a_1x + a_2x^2 + \cdots
\]

\section{Genauigkeit abschätzen mit Leibniz-Kriterium}
Wenn alternierende Reihe gegeben ist, und eine Fehlerabschätzung (z.B. \(10^{-2}\)) gesucht ist, dann kann nach dem Glied aufgehört werden, das einen Wert (im Beispiel) \(\leq \left| \frac{1}{10} \right|\).\\
\textbf{Achtung: }Funktioniert nur bei alternierenden Reihen!

\section{Aufgabenstellung \glqq Entwickeln Sie eine Reihe aus der Funktion \(f(x)\) bis zum \(n\)-ten Grad\grqq}
\begin{enumerate}
\item Erkenne elementare Reihen in der gegebenen Funktion
\item Substituiere \(x\) und ersetze \(x\) der elementaren Reihen
\end{enumerate}

\textbf{Beispiel:}

Gegeben: \(f(x) = e^{-x} - 1\)

\section{Konvergenzradius einer zusammengesetzten Reihe}
Bei einer Funktion, die aus mehreren Potenzreihen zusammengesetzt ist, gilt jeweils der \textbf{kleinste Konvergenzradius} als Gesamtkonvergenzradius der Funktion.

\section{Taylorreihe}

Eine Taylorreihe an der Stelle \(x_0\) ist definiert durch

\[
  \sum_{k=0}^{\infty}\frac{f^{(k)}(x_0)}{k!}{(x-x_0)}^k \quad \text{, also} \quad T(x)=f(x_0)+\frac{f(x_0)^{'}}{1!}{(x-x_0)}+\frac{f(x_0)^{''}}{2!}{(x-x_0)}^2+\frac{f(x_0)^{'''}}{3!}{(x-x_0)}^3+ \cdots
\]

Das \textbf{Taylorpolynom} vom Grad \(n\) wäre dementsprechend definiert durch

\[
  T_n(x)=f(x_0)+\frac{f(x_0)^{'}}{1!}{(x-x_0)}+\frac{f(x_0)^{''}}{2!}{(x-x_0)}^2+\frac{f(x_0)^{'''}}{3!}{(x-x_0)}^3+ \cdots + \frac{f(x_0)^{n}}{n!}{(x-x_0)}^n
\]

\section{Die wichtigen Reihen}
Geometrische Reihe: \hfill
\(
\sum_{k=0}^{\infty} x^k \quad = 1 + x + x^2 + x^3+ \dots + x^n \quad = \quad \frac{1}{1-q} \qquad , r = 1
\)

Reihe der \(e\)-Funktion: \hfill
\(
\sum_{k=0}^{\infty}\frac{x^k}{k!} \quad = 1 + \frac{x}{1!} + \frac{x^2}{2!} + \frac{x^3}{3!} \dots \frac{x^n}{n!} \qquad , r = \infty
\)

Sinus: \hfill
\(
\sum_{k=0}^{\infty}(-1)^k\frac{x^{2k+1}}{(2k+1)!} \quad = \frac{x}{1!} - \frac{x^3}{3!} + \frac{x^5}{5!} \dots \qquad r = \infty
\)

Cosinus: \hfill
\(
\sum_{k=0}^{\infty}(-1)^k\frac{x^{2k}}{(2k)!} \quad = 1 - \frac{x^2}{2!} + \frac{x^4}{4!} \dots \qquad r = \infty
\)

\clearpage

\setcounter{chapter}{15}

\chapter{Fourier-Reihen}

\section{Darstellung einer Fourier-Reihe}
Eine Fourier Reihe wird dargestellt durch
\[
  f(t) = \frac{a_0}{2} + \sum_{k=1}^{\infty}(a_k \cos (k \omega t) + b_k \sin (k \omega t)), \quad \omega = \frac{2 \pi}{T}
\]

\section{Berechnung von \(\frac{a_0}{2}\)}

\(\frac{a_0}{2}\) heißt auch \textbf{Mittelwert} oder \textbf{Gleichanteil}. Er entspricht dabei jeweils dem Integral der Funktion pro Periode.

\[
  \frac{a_0}{2} \quad = \quad \frac{\text{Integral über eine Periode}}{\text{Periodendauer}}
\]

\section{Berechnung der reellen Fourier-Koeffizienten \(a_k\) und \(b_k\)}

Hierbei muss die Periodendauer \(T\) und die Kreisfrequenz \(\omega = \frac{2\pi}{T}\) bekannt sein

\begin{align*}
  a_k &= \frac{2}{T} \int_{-\frac{T}{2}}^{\frac{T}{2}}f(t) \cos (k \omega t) dt\\[10pt]
  b_k &= \frac{2}{T} \int_{-\frac{T}{2}}^{\frac{T}{2}}f(t) \sin (k \omega t) dt
\end{align*}

Der Trick hierbei ist \(f(t)\) in eine Abschnittsweise definierte Funktion aufzuspalten und die Integrale dann getrennt zu berechnen.\\
Es kann zudem immer auch das Integral von \(0\) bis \(T\) berechnet werden; diese Form ist äquivalent zu \(-\frac{T}{2}\) bis \(\frac{T}{2}\)

\section{Umformung der Kosinus- und Sinusterme bei Berechnung der reellen Fourier-Koeffizienten}

\begin{empheq}[box = \fbox]{align*}
  \sin(k\pi) &= 0\\
  \cos(k\pi) &=
  \begin{cases}
    \phantom{-}1 \quad \text{für } k \text{ gerade}\\
    -1 \quad \text{für } k \text{ ungerade}\\
  \end{cases}\\
  &\textcolor{myred}{\rightarrow (-1)^k}\\
  - \cos(k\pi) &=
  \begin{cases}
    -1 \quad \text{für } k \text{ gerade}\\
    \phantom{-}1 \quad \text{für } k \text{ ungerade}\\
  \end{cases}\\
  &\textcolor{myred}{\rightarrow (-1)^{k+1}}\\[1em]
  \cos(2\pi k) &= 1
\end{empheq}

\section{Stetigkeit und Koeffizienten}

Wenn die Koeffizienten der Fourierreihe proportional zu \(\frac{1}{k}\) sind ist die Reihe unstetig
\myhspace \mytextred{$\rightarrow$ Langsame Konvergenz}

Wenn die Koeffizienten dagegen proportional zu \(\frac{1}{k^2}\) sind ist die Reihe stetig
\myhspace \mytextred{$\rightarrow$ Schnelle Konvergenz}

\section{Darstellung einer Funktion vom Grad \(\mathbf{n}\)}
Ist die Fourier-Reihe einer Funktion vom Grad \(n\) gesucht, so ist nach der Fourier-Reihe

\[
  p_n(t) = \frac{a_0}{2} + \sum_{k=1}^{n}(a_k \cos (k \omega t) + b_k \sin (k \omega t)), \quad \omega = \frac{2 \pi}{T}
\]

gefragt.\\[1em]
Es müssen hier also lediglich die Fourier-Koeffizienten \(a_1, a_2, \cdots a_n\) bzw. \(b_1, b_2, \cdots b_n\) berechnet werden.

\section{Komplexe Fourier-Reihe}

\subsection{Komplexer Fourier-Koeffizient \(\mathbf{c_k}\)}
Der komplexe Fourier-Koeffizient \(c_k\) berechnet sich durch die Formel

\[
  c_k = \frac{1}{T} * \int_0^T f(t)e^{-ikwt}dt, \quad k = 0,\ \pm 1,\ \pm 2, \cdots
\]

\section{Umrechnung der \(e\)-Terme beim Berechnen von \(c_k\)}
\begin{align*}
  e^{-ik2\pi} &= 1\\
  e^{-ik\pi} &= (-1)^k\\[1em]
  \rightarrow \qquad &1 \text{ für } k \text{ gerade}\\
  -&1 \text{ für } k \text{ ungerade}
\end{align*}

\section{Umrechnung von \(c_k\) zu \(a_k\) und \(b_k\)}
\begin{empheq}[box = \fbox]{align*}
  a_k &= 2 \mathrm{Re}(c_k)\\[1em]
  b_k &= -2 \mathrm{Im}(c_k)\\[1em]
  \rightarrow c_k &= \frac{a_k-ib_k}{2}
\end{empheq}


\section{Generelle Vorgehensweise beim Erstellen einer Fourier-Reihe}
\begin{figure}[h]
  \begin{enumerate}
  \item Fourierreihe skizzieren
  \item \(T\) und \(\omega\) ablesen / berechnen
  \item Feststellen ob die Fourierreihe gerade, ungerade oder keins von beidem ist
  \item Fourierkoeffizienten berechnen
  \item Falls Fourierkoeffizienten komplex: Umrechnen in reelle Darstellung
  \item Erste Reihenglieder der Fourierreihe berechnen und aufschreiben
  \end{enumerate}
\end{figure}


\begin{figure}[H]
\section{Integrale von \(t*\sin(k \omega t)\) und \(t*\cos(k \omega t)\)}

\begin{empheq}[box = \fbox]{align*}
  \int{t * \sin(k \omega t)}\ dt &= \frac{\sin(k \omega t) - k \omega t \cos(k \omega t)}{k^2 \omega^2}\\[1em]
  \int{t * \cos(k \omega t)}\ dt &= \frac{k \omega t \sin(k \omega t) + \cos(k \omega t)}{k^2 \omega^2}
\end{empheq}

Im Falle von \(\omega = \frac{2 \pi}{T}\ ;\ \frac{2 \pi}{2 \pi}\) verschwindet der \(\omega\)-Anteil.
\end{figure}



\chapter{Wichtige Integrale}

\paragraph{\(ln(x)\)} \hfill
\(
\int \frac{1}{x}
\)

\paragraph{\(arctan(x)\)} \hfill \(\int \frac{1}{1+x^2}\)



\end{document}

%%% TeX-command-extra-options: "-shell-escape"
%%% Local Variables:
%%% mode: latex
%%% TeX-master: t
%%% End: